\documentclass{ludis} % pieejams https://github.com/rihardsk/LU-nosl-guma-darbs---LaTeX

% xelatex
\usepackage{fontspec}
\usepackage{xunicode}
\usepackage{xltxtra}

%\usepackage[utf8]{inputenc}

\usepackage[]{hyperref}
\hypersetup{
    colorlinks=false
}
\urlstyle{same}

% languages
\usepackage{fixlatvian}
\usepackage{polyglossia}
\setdefaultlanguage{latvian}
\setotherlanguages{english,russian}

% fonts
\usepackage{xltxtra}
% \setmainfont[Mapping=tex-text]{Times New Roman}
%\defaultfontfeatures{Scale=MatchLowercase,Mapping=tex-text}

% bibliography
%\usepackage{csquotes}
\usepackage[
    backend=biber,
    style=numeric-comp,
    sorting=none,
    natbib=true,
    url=false,
    doi=true%,
    %eprint=false
]{biblatex}
\addbibresource{bibliography.bib}

% toc
\setcounter{secnumdepth}{3}
\setcounter{tocdepth}{3}

%tables
\usepackage{longtable}

%papildus matemātika
\usepackage[showonlyrefs]{mathtools}
\newenvironment{thmenum}
 {\begin{enumerate}[label=\upshape(\arabic*),ref=\thethm(\arabic*)]}
 {\end{enumerate}}
\usepackage{amsmath}

%pseidokodam
%\usepackage[noend]{algpseudocode}
%\usepackage{algpseudocode}
\usepackage[boxed,linesnumbered]{algorithm2e}
\SetAlgorithmName{Algoritms}{}{Algoritmu saraksts}

%main line spacing, kur vajag
\usepackage{setspace}

%lai flushotu floatus
\usepackage{placeins}

%images
\usepackage{graphicx}
\usepackage{float}

%dalīšana kolonnās
\usepackage{multicol}

%saraksti
\usepackage{enumitem}

\fakultate{Datorikas}
\nosaukums{Paredzošā stimulētā mācišanās}
\darbaveids{Maģistra kursa}
\autors{Rihards Krišlauks}
\studapl{rk09006}
\vaditajs{Asoc.prof., Dr. dat. Jānis Zuters}
%\recenzents{Juris Vīksna profesors Dr.sc.comp.}
\vieta{Rīga}
\gads{2015}

\begin{document}
\maketitle

\begin{abstract-lv}
  TODO

\keywords{stimulētā mācīšanās; neironu tīkli; Markova izvēles procesi; nepārtrauktas telpas.}
\end{abstract-lv}
\clearpage

\begin{abstract-en}
  TODO

\keywords{reinforcement learning; artificial neural networks; Markov decision processes; continuous spaces.}
\end{abstract-en}


\tableofcontents

\iffalse
\specnodala{Apzīmējumu saraksts}
\setlength\LTleft{0pt}
\setlength\LTright{0pt}
\begin{longtable}{| c | p{28em} |}
  \hline
  \textbf{Apzīmējums} & \textbf{Atšifrējums}\\ 
  \endhead

  \hline
  $D_X \in \mathbb{N}_+$ & \\ %TODO šis jādefinē arī telpas elementiem
  $S \subseteq \mathbb{R}^{D_S}$ & \\
  $A \subseteq \mathbb{R}^{D_A}$ & \\
  $R:S \times A \times S \rightarrow \mathbb{R}$ & \\
  $T:S \times A \times S \rightarrow [0,1]$ & Apzīmējuma nosaukums \\
  $\pi(s, a)$ &  Apzīmējuma nosaukums 2\\
  \hline
\end{longtable}
\fi

\specnodala{Ievads}
Pēdējā laikā teorētiskajā neirozinātnē plašu piekrišanu sāk iegūt t.s.
paredzošās kodēšanas (predictive coding) teorija, kuras pamattēze ir --
smadzenes ir paredzēšanas mašīnas, tās pastāvīgi cenšas sapārot ienākošos
sensoru signālus ar augsta līmeņa paredzējumiem ar mērķi minimizēt paredzēšanas
kļūdas. Paredzošā kodēšana sniedz vienotu skatījumu uz procesiem, kas ir pamatā
cilvēka apziņai, un piedāvā mehānismu, kas ļauj izskaidrot plašu klāstu ar
neirozinātnē pētītiem cilvēka apziņas fenomeniem \autocite{Clark2013}.

Ideja par smadzenēm kā māšīnām, kas, ņemot vērā pašreizējos novērojumus, cenšas
paredzēt nākotnes novērojumus, ir ļoti pievilcīga no dažādu mašīnmācīšanās
paradigmu skatpunkta. Šis vienkāršais mehānisms šķiet viegli formulējams kā
mašīnmācīšanās problēma, un šķiet it īpaši piemērots, lai to izteiktu kā
stimulētās mācīšanās problēmu.

Stimulētā mācīšanās ir mašīnmācīšanās paradigma, kas formalizē mācīšanos kā
mašīnmācīšanās uzdevumu. Tās pamatā ir ideja par vidi un aģentu, kas tajā
darbojas. Aģenta mērķis ir, ņemot vērā ārēju atalgojuma signālu, veikt darbības
vidē tā, lai maksimizētu saņemto atalgojumu. Šis vienkāršais, bet spēcīgais,
uzstādījums ļauj aprakstīt un risināt ļoti plašu problēmu loku, kur optimālā
rīcības stratēģija nav zināma, bet ir iespējams nodefinēt stāvokli, ko aģentam
būtu jācešas sasniegt. Aģenta ziņā paliek, darbojoties vidē, laika gaitā atrast
optimālo stratēģiju, kas ļautu sasniegt pēc iespējas lielāku atalgojumu. Kā
piemērus šādi risināmām problēmām var minēt orientēšanos labirintā vai
automātisku lidaparāta kontroli.

No augstāk minētā dabīgi seko doma par paredzošās kodēšanas ideju pielietošanu
stimulētās mācīšanās uzdevumu risināšanā. Interesi raisa ideja par vides nākamā
stāvokļa paredzēšanu, ņēmot vērā pašreizējo, un vai tas palīdzētu mācīšanās
procesā. Tas liktu stimulētās mācīšanās aģentam tuvināti iemācīties vides
modeli. Tuvākas izpētes vērts ir jautājums, vai šāds iekšējs modelis palīdzētu
macīšanās procesā.

Šajā darbā tiks pētītas iespējas, izmantot paredzošās kodēšanas idejas stimulētās
mācīšanas uzdevumu risināšanai. Darbs tiek iesākts ar vispārīgu stimulētās
mācīšanās paradigmas apskatu, kur autors galvenokārt pieturas pie izklāsta kas
atrodams \autocite{Krislauks2015}. Tam seko COMBO CACLA algoritma apraksts, un
salīdzinājums ar CACLA algoritmu, uz kā tas ir balstīts. %TODO papildināt, kad
                                %ir vairāk gatavs no satura

\chapter{Nodaļās nosaukums} \label{chap:mdp}
TODO
\section{Apakšnodaļas nosaukums}
TODO

\chapter{Diskusija}
TODO
\chapter{Secinājumi}
TODO


\printbibliography
\end{document}

%%% Local Variables:
%%% coding: utf-8
%%% mode: latex
%%% TeX-master: t
%%% TeX-engine: xetex
%%% reftex-default-bibliography: ("bibliography.bib")
%%% End:

%%% TeX-command-extra-options: "-shell-escape"