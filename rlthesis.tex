\documentclass{ludis} % pieejams https://github.com/rihardsk/LU-nosl-guma-darbs---LaTeX

% xelatex
\usepackage{fontspec}
\usepackage{xunicode}
\usepackage{xltxtra}

%\usepackage[utf8]{inputenc}

\usepackage[]{hyperref}
\hypersetup{
    colorlinks=false
}
\urlstyle{same}

% languages
\usepackage{fixlatvian}
\usepackage{polyglossia}
\setdefaultlanguage{latvian}
\setotherlanguages{english,russian}

% fonts
\usepackage{xltxtra}
% \setmainfont[Mapping=tex-text]{Times New Roman}
%\defaultfontfeatures{Scale=MatchLowercase,Mapping=tex-text}

% bibliography
%\usepackage{csquotes}
\usepackage[
    backend=biber,
    style=numeric-comp,
    sorting=none,
    natbib=true,
    url=false,
    doi=true%,
    %eprint=false
]{biblatex}
\addbibresource{bibliography.bib}

% toc
\setcounter{secnumdepth}{3}
\setcounter{tocdepth}{3}

%tables
\usepackage{longtable}

%papildus matemātika
\usepackage[showonlyrefs]{mathtools}
\newenvironment{thmenum}
 {\begin{enumerate}[label=\upshape(\arabic*),ref=\thethm(\arabic*)]}
 {\end{enumerate}}
\usepackage{amsmath}

%pseidokodam
%\usepackage[noend]{algpseudocode}
%\usepackage{algpseudocode}
\usepackage[boxed,linesnumbered]{algorithm2e}
\SetAlgorithmName{Algoritms}{}{Algoritmu saraksts}

%main line spacing, kur vajag
\usepackage{setspace}

%lai flushotu floatus
\usepackage{placeins}

%images
\usepackage{graphicx}
\usepackage{float}

%dalīšana kolonnās
\usepackage{multicol}

%saraksti
\usepackage{enumitem}

\fakultate{Datorikas}
\nosaukums{Paredzošā stimulētā mācišanās}
\darbaveids{Maģistra kursa}
\autors{Rihards Krišlauks}
\studapl{rk09006}
\vaditajs{Asoc.prof., Dr. dat. Jānis Zuters}
%\recenzents{Juris Vīksna profesors Dr.sc.comp.}
\vieta{Rīga}
\gads{2015}

\begin{document}
\maketitle

\begin{abstract-lv}
  TODO

\keywords{stimulētā mācīšanās; neironu tīkli; Markova izvēles procesi; nepārtrauktas telpas.}
\end{abstract-lv}
\clearpage

\begin{abstract-en}
  TODO

\keywords{reinforcement learning; artificial neural networks; Markov decision processes; continuous spaces.}
\end{abstract-en}


\tableofcontents

\iffalse
\specnodala{Apzīmējumu saraksts}
\setlength\LTleft{0pt}
\setlength\LTright{0pt}
\begin{longtable}{| c | p{28em} |}
  \hline
  \textbf{Apzīmējums} & \textbf{Atšifrējums}\\ 
  \endhead

  \hline
  $D_X \in \mathbb{N}_+$ & \\ %TODO šis jādefinē arī telpas elementiem
  $S \subseteq \mathbb{R}^{D_S}$ & \\
  $A \subseteq \mathbb{R}^{D_A}$ & \\
  $R:S \times A \times S \rightarrow \mathbb{R}$ & \\
  $T:S \times A \times S \rightarrow [0,1]$ & Apzīmējuma nosaukums \\
  $\pi(s, a)$ &  Apzīmējuma nosaukums 2\\
  \hline
\end{longtable}
\fi

\specnodala{Ievads}
  TODO

\chapter{Nodaļās nosaukums} \label{chap:mdp}
TODO
\section{Apakšnodaļas nosaukums}
TODO

\chapter{Diskusija}
TODO
\chapter{Secinājumi}
TODO


\printbibliography
\end{document}

%%% Local Variables:
%%% coding: utf-8
%%% mode: latex
%%% TeX-master: t
%%% TeX-command-extra-options: "-shell-escape"
%%% TeX-engine: xetex
%%% End:
